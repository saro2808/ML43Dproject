\begin{abstract}
We explore the use of large Vision-Language Models (VLMs) for the task of cross-view 2D instance mask grouping,
aiming to associate object masks observed in multiple camera views into consistent 3D instances. Using the Replica
indoor dataset, we formulate mask grouping as a binary visual reasoning problem: given two cropped instance masks
from different views, determine whether they correspond to the same underlying 3D object. We establish a baseline
using the pretrained Qwen3-VL model and subsequently fine-tune it using LoRA adapters, without modifying the base
model weights. Our experiments on held-out scenes demonstrate that LoRA fine-tuning improves mask ID consistency
highlighting the potential of VLMs without explicit 3D reasoning at inference time.
The codebase is available at \url{https://github.com/saro2808/ML43Dproject}.
\end{abstract}

\section{Introduction}

Instance-level scene understanding is a core challenge in 3D perception \cite{Qi_2017_CVPR}. While modern pipelines
often rely on explicit geometric reasoning — such as point cloud clustering or hierarchical feature aggregation methods
(e.g., PointNet++ \cite{qi2017pointnetdeephierarchicalfeature}) — these approaches typically require dense reconstruction
and carefully designed heuristics \cite{schult2023mask3dmasktransformer3d}. In contrast, recent Vision-Language Models (VLMs)
have shown remarkable zero-shot reasoning abilities across images \cite{pmlr-v139-radford21a, kirillov2023segment},
suggesting they may implicitly capture object identity cues such as shape, texture, and semantic consistency
\cite{oquab2024dinov, zhang-etal-2025-vlm2}.

In this work, we investigate whether a VLM can be trained to group 2D instance masks across views into consistent 3D
object identities. Instead of predicting explicit correspondences or embeddings, we frame the problem as a binary
question-answering task: Do these two image crops correspond to the same object?

Our contributions are:
\begin{enumerate}
    \item a VLM-compatible formulation of multi-view 2D mask grouping using the Replica dataset,
    \item a baseline evaluation of pretrained Qwen3-VL~\cite{bai2025qwen3vltechnicalreport} on this task,
    \item a lightweight LoRA fine-tuning strategy that improves consistency without updating base weights.
    % \item a detailed evaluation protocol that accounts for ambiguous VLM outputs.
\end{enumerate}


\section{Dataset and Preprocessing}

\subsection{Replica Dataset}

We use the Replica dataset, a photorealistic indoor dataset with high-quality meshes, camera trajectories,
and per-vertex instance annotations. Replica provides an ideal testbed for studying multi-view instance consistency
under controlled conditions.

Eight scenes are available: 5 offices and 3 rooms. Given that object distributions do not vary drastically across scenes
(see~Figure~\ref{fig:scene_instance_distribution}) we split the dataset as follows:
\begin{itemize}
    \item 6 training scenes (4 offices and 2 rooms): used for LoRA fine-tuning,
    \item 2 evaluation scenes (1 office and 1 room): office4 and room2.
\end{itemize}

Inspecting Figure~\ref{fig:scene_instance_distribution} one may observe that the distributions of office4 and room2
are pretty much in the middle of the others, which justifies their usage as evaluation scenes.

\begin{figure}[htbp]
    \centering
    \pgfplotsset{cycle list/Set1-8} % Add this line here
\begin{tikzpicture}
    \begin{axis}[
        width=\linewidth, % Scaled to the container in main.tex
        height=7cm,
        ymode=log,
        log origin=infty,          % Helps with log scale boundaries
        filter discard warning=false, % Suppresses warnings about 0s
        unbounded coords=discard,  % Skips values that can't be plotted
        xlabel={Instance Rank (by size)},
        ylabel={Number of Vertices},
        grid=major,
        grid style={dashed, gray!30},
        legend pos=north east,
        legend style={font=\tiny, cells={anchor=west}},
        legend columns=2,
        cycle list name=Set1-8, % Uses the 'Set1' palette with 8 distinct colors
        % cycle list name=colorbrewer-qualitative,
        no markers,
        thick,
        xmin=0,
    ]
        % Automation loop to plot all 8 scenes
        \foreach \s in {office0, office1, office2, office3, office4, room0, room1, room2} {
            \addplot table [col sep=comma, x=rank, y=size] {diagrams/data/plot_data_\s.csv};
            \addlegendentryexpanded{\s}
        }
    \end{axis}
\end{tikzpicture}
    \caption{Distribution of instance sizes (vertex counts) across the Replica dataset. The logarithmic scale reveals a consistent long-tail distribution across both office and room environments.}
    \label{fig:scene_instance_distribution}
\end{figure}

\subsection{Multi-View Rendering and Instance Mask Generation}

For each scene, we render multi-view RGB images and instance masks using PyTorch3D. Each view is generated from the original camera poses and intrinsics provided by Replica.

Key steps:
\begin{itemize}
    \item load the textured mesh and camera parameters,
    \item render RGB images using a flat shader,
    \item render instance masks by mapping rendered faces to ground-truth instance IDs via majority voting over vertex
    labels (see~Figure~\ref{fig:pixel_mapping}).
\end{itemize}

\begin{figure}[htbp]
    \centering
    \includestandalone[width=0.5\textwidth]{diagrams/mask_creation} % no extension needed
    \caption{Hierarchical mapping from 2D pixels to 3D instance labels using renderer metadata and majority voting.}
    \label{fig:pixel_mapping}
\end{figure}

Each rendered view is saved as:
\begin{itemize}
    \item \texttt{rgb.png}: the RGB image,
    \item \texttt{instance\_mask.npy}: per-pixel instance IDs,
    \item \texttt{unique\_instances.npy}: list of visible instance IDs.
\end{itemize}

\begin{figure}[htbp]
    \centering
    \includegraphics[width=0.5\textwidth]{images/rgb_and_mask.png}
    \caption{Visualization of the RGB image and instance mask.}
    \label{fig:rgb_mask}
\end{figure}

\subsection{Instance Pair Construction}

To train and evaluate the VLM, we construct pairs of instance crops:
\begin{itemize}
    \item two views are sampled;
    \item a shared instance ID defines a positive pair,
    \item a different instance ID defines a negative pair.
\end{itemize}

When constructing the dataset for a given scene we build a map giving for each instance ID the view pairs containing
that instance. However, to combat instance imbalance we bound the number of view pairs for each instance by 10.
Moreover, for the same reason dataset items are sampled uniformly over instance IDs.

Each instance crop is extracted by tightly bounding the corresponding mask region. We add a small padding of width
10 pixels to make the crops less abrupt and provide the VLM with more local context around the object boundaries.
The resulting pair is packaged into a VLM-style prompt whether they refer to the same object in the 3D scene
(see~Figure~\ref{fig:pipeline}).
This formulation removes the need for explicit geometric reasoning during inference.

\begin{figure*}[t] % Use [t] for top of page; [h] doesn't work well with starred figures
    \centering
    \begin{tikzpicture}[
    node distance=1.2cm and 0.8cm, 
    every node/.style={font=\small},
    img/.style={draw, rounded corners, inner sep=0pt, fill=white},
    box/.style={draw, rounded corners, minimum width=2.0cm, minimum height=1.0cm, align=center, fill=white},
    arrow/.style={->, thick}
]

% --- 1. Define View Anchors First (to establish vertical center) ---
\node[img] (mask1)
    {\includegraphics[width=3.2cm]{images/pipeline/mask_office4_view14.png}};
    
\node[img, below=1.8cm of mask1] (mask2)
    {\includegraphics[width=3.2cm]{images/pipeline/mask_office4_view64.png}};

% --- 2. 3D scene (Now centered vertically between mask1 and mask2) ---
% We use 1.2cm to give room for the RGB images shifting left
\node[box, left=1.2cm of $(mask1.west)!0.5!(mask2.west)$] (scene) {Replica 3D\\(mesh)};

% --- 3. View A Stack ---
\node[img] (rgb1) at ([xshift=-0.7cm, yshift=0.7cm]mask1.center)
    {\includegraphics[width=3.2cm]{images/pipeline/rgb_office4_view14.png}};
\node[above=0.2cm of rgb1] {\textbf{View A}};

% --- 4. View B Stack ---
\node[img] (rgb2) at ([xshift=-0.7cm, yshift=0.7cm]mask2.center)
    {\includegraphics[width=3.2cm]{images/pipeline/rgb_office4_view64.png}};
\node[above=0.2cm of rgb2] {\textbf{View B}};

% --- 5. Crops ---
\node[img, right=1.2cm of rgb1] (crop1)
    {\includegraphics[width=2.0cm]{images/pipeline/crop_office4_view14_object20017.png}};
\node[img, right=1.2cm of rgb2] (crop2)
    {\includegraphics[width=2.0cm]{images/pipeline/crop_office4_view64_object20017.png}};

% --- 6. VLM box ---
\node[box, right=0.8cm of $(crop1.east)!0.5!(crop2.east)$, minimum width=2.2cm] (vlm) {
    \textbf{VLM}\\[-2pt]
    \scriptsize Qwen3-VL\\
    \scriptsize ``Same object?''
};

% --- 7. Output ---
\node[box, right=0.6cm of vlm, minimum width=1.5cm] (out) {\textbf{Yes/No}};

% --- Arrows ---
% Arrows now originate from the scene and branch out symmetrically
\draw[arrow] (scene.east) -- (rgb1.west); 
\draw[arrow] (scene.east) -- (rgb2.west);

\draw[arrow] (rgb1.east) -- (crop1.west);
\draw[arrow] (rgb2.east) -- (crop2.west);

\draw[arrow] (crop1.east) -- (vlm.160);
\draw[arrow] (crop2.east) -- (vlm.200);

\draw[arrow] (vlm) -- (out);

\end{tikzpicture} % Removed resizebox to let the figure scale naturally or use a fixed width
    \caption{Overview of the proposed pipeline. Multi-view RGB images and instance masks are rendered from a 3D scene, cropped into instance-level regions, and evaluated by a Vision-Language Model via a binary query.}
    \label{fig:pipeline}
\end{figure*}


\section{Model and Training}

\subsection{Base Model: Qwen3-VL}

We use Qwen3-VL-8B, the Qwen variant with 8B total parameters, a large Vision-Language Model capable of
multi-image reasoning. The base model is evaluated without any fine-tuning to establish a baseline for zero-shot performance.

\subsection{LoRA Fine-Tuning}

To adapt the model to our task efficiently, we fine-tune Qwen3-VL using Low-Rank Adaptation (LoRA) \cite{hu2022lora}:
\begin{itemize}
    \item only LoRA adapters are trained; base model weights remain frozen;
    \item vision, language, attention, and MLP modules are adapted;
    \item 4 bit quantization is utilized.
\end{itemize}

This choice is motivated by:
\begin{itemize}
    \item limited GPU memory (Colab environment, one Tesla T4 GPU with 15GB RAM),
    \item desire to preserve general VLM capabilities,
    \item faster training and reduced storage requirements.
\end{itemize}

Training is performed for 2 epochs with 1000 samples from each train scene, which we found sufficient
to observe measurable improvements.

\begin{table}[htbp]
    \centering
    \small
    
    % --- First Table ---
    \caption{LoRA Hyperparameters}
    \label{tab:lora_params}
    \begin{tabular}{l c}
        \toprule
        \textbf{Parameter} & \textbf{Value} \\
        \midrule
        LoRA rank ($r$) & 16 \\
        LoRA alpha & 16 \\
        LoRA dropout & 0.05 \\
        Learning rate & $2 \times 10^{-4}$ \\
        Batch size & 2 \\
        Grad. accum. & 4 \\
        Epochs & 2 \\
        FP16 & Yes \\
        \bottomrule
    \end{tabular}

    \vspace{1.5em} % Vertical gap between the two distinct tables

    % --- Second Table ---
    \caption{Dataset Configuration}
    \label{tab:dataset_cfg}
    \begin{tabular}{l c}
        \toprule
        \textbf{Parameter} & \textbf{Value} \\
        \midrule
        Max train samples & $6\times 1000$ \\
        Max eval samples & $2\times 300$ \\
        Neg. pair prob. & 0.5 \\
        \bottomrule
    \end{tabular}
\end{table}

\subsection{Training Stability and Checkpointing}

Given the constraints of Colab (session crashes, limited disk space), we:
\begin{itemize}
    \item periodically save checkpoints to Google Drive,
    \item overwrite the previous checkpoint to conserve storage,
    \item train for a fixed small number of epochs rather than early stopping.
\end{itemize}

While intermediate evaluation during training could be added, the limited training duration makes validation
loops less critical in this setting.


\section{Evaluation Protocol}

\subsection{Metrics}

We evaluate performance by tracking accuracy, precision, recall and ambiguity rate (fraction of samples where
the model fails to clearly answer “Yes” or “No”). Ambiguous outputs are handled explicitly:
\begin{itemize}
    \item attempt to parse the model's response;
    \item retry with a forced instruction (``Answer with exactly one word: Yes or No.");
    \item if still ambiguous, count the sample as ambiguous and force an incorrect prediction.
\end{itemize}

This protocol avoids artificially inflated accuracy due to skipped samples.

Remarkably, phrasing the problem as binary classification makes evaluation much easier than if we were
to give more than two crops to the VLM and ask to group them (a substantial difficulty could arise from parsing
VLM's output).

\subsection{Scenes and Setup}

Evaluation is conducted on the scenes \textbf{office4} and \textbf{room2}. For each scene,
300 instance pairs are evaluated with resume support, ensuring robustness to interruptions.

\begin{table}[htbp]
    \centering
    \small
    \setlength{\tabcolsep}{4pt} % Slightly tighten the space between columns
    \caption{Performance Comparison. Acc: Accuracy, Prec: Precision, Rec: Recall, Amb.: Ambiguity Rate.}
    \label{tab:model_performance}
    \begin{tabular}{llcccc}
        \toprule
        \textbf{Scene} & \textbf{Model} & \textbf{Acc.} & \textbf{Prec.} & \textbf{Rec.} & \textbf{Amb.} \\
        \midrule
        office4 & Base & 0.74 & 0.67 & 0.88 & 0.00 \\
        office4 & LoRA & 0.82 & 0.76 & 0.92 & 0.00 \\
        \addlinespace
        room2   & Base & 0.80 & 0.76 & 0.91 & 0.00 \\
        room2   & LoRA & 0.92 & 0.91 & 0.94 & 0.00 \\
        \bottomrule
    \end{tabular}
\end{table}


\section{Results and Discussion}

\subsection{Baseline Performance}

The base Qwen3-VL model demonstrates acceptably good performance, suggesting that large VLMs encode rich
visual priors about object identity. Surprisingly, the model is quite assertive in its answers — we observe
zero ambiguous responses (possibly after retrying with more forceful prompt). This attests to the model's
high confidence and lack of hedge-word bias, even when its predictions are technically incorrect.
However, there is clearly room for improvement, especially with hard negatives.

\subsection{Effect of LoRA Fine-Tuning}

After LoRA fine-tuning:
\begin{itemize}
    \item accuracy and precision improve considerably, especially for negative pairs,
    \item recall improves mildly (or even worsens, see~Table~\ref{tab:hyperparam_tuning}).
\end{itemize}

This suggests that LoRA fine-tuning helps the model internalize task-specific cues, such as instance-level
shape consistency across views. Particularly, as we could anticipate, after fine-tuning too (similar to the base model)
when given the prompt Qwen3-VL is firm in its answers (again, possibly after the forceful second prompt); there
is no any ambiguity whatsoever.

\begin{table}[htbp]
    \centering
    \caption{Hyperparameter tuning results (Rank and LR).}
    \label{tab:hyperparam_tuning}
    \small
    \setlength{\tabcolsep}{3.5pt} % Tighten column spacing
    \begin{tabular}{l l l c c c c}
        \toprule
        \textbf{R} & \textbf{LR} & \textbf{Scene} & \textbf{Acc.} & \textbf{Prec.} & \textbf{Rec.} & \textbf{Amb.} \\
        \midrule
         8 & $2e^{-4}$ & office4 & 0.82 & 0.75 & 0.92 & 0.00 \\
           &           & room2   & 0.92 & 0.91 & 0.94 & 0.00 \\
         \addlinespace
         8 & $2e^{-5}$ & office4 & 0.82 & 0.83 & 0.77 & 0.00 \\
           &           & room2   & 0.89 & 0.92 & 0.87 & 0.00 \\
        \midrule
        16 & $2e^{-4}$ & office4 & 0.82 & 0.76 & 0.92 & 0.00 \\
           &           & room2   & 0.92 & 0.91 & 0.94 & 0.00 \\
        \midrule
        32 & $2e^{-4}$ & office4 & 0.84 & 0.82 & 0.84 & 0.00 \\
           &           & room2   & 0.91 & 0.90 & 0.93 & 0.00 \\
        \midrule
        64 & $2e^{-4}$ & office4 & 0.85 & 0.83 & 0.85 & 0.00 \\
           &           & room2   & 0.92 & 0.93 & 0.91 & 0.00 \\
        \addlinespace
        64 & $2e^{-5}$ & office4 & 0.82 & 0.88 & 0.72 & 0.00 \\
           &           & room2   & 0.89 & 0.95 & 0.83 & 0.00 \\
        \bottomrule
    \end{tabular}
\end{table}


\subsection{Failure Modes}

Common failure cases include:
\begin{itemize}
    \item symmetric or repetitive objects (e.g. chairs, walls),
    \item severe occlusions,
    \item very small instance crops with limited visual context.
\end{itemize}

Another thing worth mentioning is the model's bias toward answering ``Yes" for difficult negative pairs
(see~Table~\ref{tab:negative_pair_bias}):

\begin{table}[htbp]
    \centering % This only works if the table is narrower than the column
    \small
    \caption{Mistake Rate on Negative Pairs (Rank-16)}
    \label{tab:negative_pair_bias}
    \begin{tabular}{llccc} % Changed from tabularx to tabular
        \toprule
        \textbf{Scene} & \textbf{Model} & \textbf{Samples} & \textbf{Neg. Errors} & \textbf{Share} \\
        \midrule
        office4 & Base & 78 & 61 & 0.78 \\
        office4 & LoRA & 53 & 42 & 0.79 \\
        \addlinespace
        room2   & Base & 61 & 47 & 0.77 \\
        room2   & LoRA & 25 & 15 & 0.60 \\
        \bottomrule
    \end{tabular}
\end{table}

We present some difficult pairs in Appendix~\ref{sec:qual_vis}.


% \section{Related Work (Suggested Discussion)}

% You can briefly relate this work to:
% \begin{itemize}
%     \item Multi-view instance association in 3D reconstruction pipelines (e.g., OpenYOLO3D).
%     \item Contrastive learning for instance correspondence.
%     \item Recent work on VLMs for geometric reasoning and spatial understanding.
% \end{itemize}

% A key distinction is that our method avoids explicit 3D reasoning at inference time, relying instead on
% learned visual semantics.

\section{Conclusion}

We present a VLM-based approach for grouping 2D instance masks across views, demonstrating that large
Vision-Language Models can be adapted for geometric consistency tasks with minimal fine-tuning. Our
results show that LoRA fine-tuning improves instance grouping accuracy and other related metrics, even with
limited training data and compute. The key characteristic of our approach is that it avoids explicit 3D
reasoning at inference time, relying instead on learned visual semantics.

Future work includes:
\begin{itemize}
    \item extending beyond binary decisions to clustering,
    \item backprojecting the predicted 2D masks to reconstruct 3D instance masks.
\end{itemize}


% \printbibliography

\clearpage
\appendix
\section{Qualitative Visualizations}
\label{sec:qual_vis}

What follows are visualizations of pairs which were misclassified with the base Qwen3-VL but
discriminated correctly with the fine-tuned one.

\begin{figure*}[htbp]
    \centering
    \includegraphics[width=\textwidth]{images/hard_neg_pair_shape_0.png}
    \includegraphics[width=\textwidth]{images/hard_neg_pair_shape_1.png}
    \caption{A negative pair. Assumed difficulty: shape similarity.}
    \label{fig:hard_neg_pair_one}
\end{figure*}

\begin{figure*}[htbp]
    \centering
    \includegraphics[width=\textwidth]{images/hard_neg_pair_small_occl_0.png}
    \includegraphics[width=\textwidth]{images/hard_neg_pair_small_occl_1.png}
    \caption{A negative pair. Assumed difficulty: smallness and occlusion.}
    \label{fig:hard_neg_pair_two}
\end{figure*}

\begin{figure*}[htbp]
    \centering
    \includegraphics[width=\textwidth]{images/hard_pos_pair_chair_0.png}
    \includegraphics[width=\textwidth]{images/hard_pos_pair_chair_1.png}
    \caption{A positive pair. Assumed difficulty: symmetry and occlusion.}
    \label{fig:hard_pos_pair_one}
\end{figure*}

\FloatBarrier % This forces all figures to render BEFORE the bibliography starts